\vspace*{\subsecspace}
\section{Discussion and Future Directions}
\label{sec:discussion}

% We have made an initial exploration into constrained preemptions, and many questions remain to be answered.
% Below, we discuss some aspects of pertaining to our model and avenues for future investigation.

Constrained preemptions are a relatively unexplored phenomenon and challenging to model.
Our model and the associated data expand transient cloud computing to beyond EC2-spot.
We have evaluated the model under different practical conditions including different VM types and temporal domains, and have shown it to be general and robust. 
However, many questions and avenues of future investigation remain open:

\noindent \textbf{What if preemption characteristics change?}
%Ultimately, the preemption characteristics are based on the cloud provider policies, the supply and demand of transient and on-demand and reserved VMs, etc., and may change over time.  
Our model allows detecting policy and phase changes by comparing observed data with model-predictions and detect change-points, and 
a long-running cloud service can continuously update the model based on recent preemption behavior. 
However, changes are rare: Google's preemption policy has not changed since its inception in 2015. 
%EC2 pricing and preemptions were relatively stable~\cite{hotcloud-not-bid} until 2017~\cite{irwin-icccn19, baughman2019deconstructing, spotweb}.
%
Regardless, VMs with constrained preemptions are an interesting \emph{new} type of transient resource, and our analysis, observations, and policies should continue to be relevant. 
Furthermore, we demonstrate that the multi-phase bathtub failure distribution may be a fundamental characteristic of constrained preemptions that benefit both the cloud platform and applications, and thus models that capture the distinct preemption phases would still be relevant even if the finer-grained preemption characteristics change. % over time. 
%
We have also shown that our policies are not particularly sensitive to the model parameters, and even using a ``wrong'' or outdated model can provide significant benefits compared to existing memoryless models. 

\noindent \textbf{\emph{Phase-wise} model.}
Our statistical analysis indicates that the preemption rates have three distinct phases. 
%Our model is a continuously differentiable and allows capturing the three phases reasonably well. 
\vikram{The analytical model derived in this work is continuously differentiable and allows capturing the three phases reasonably well.}
It may be possible to use a ``phase-wise'' model such as a piece-wise continuously differentiable model, where the three phases are modeled either as segmented linear regions (found using segmented linear regression), or an initial exponential phase and two linear phases. 
Such a piece-wise model could capture the phase transitions with even more accuracy.
\vikram{Further, one may consider the use of such models (heurestics) as preferred for their simplicity. However the analytical form for the predictive model enables for a cleaner integration with the resource management service. The analytical form for the preemption distributions as well as the rates can also be used to distill contributions of distinct phases in a more convenient manner. It also provides a measure to distill the contirbutions of the discontinuties present in the real operation (which are coarse-grained over in the continuously differentiable model) in changing the VM expected lifetimes and checkpointing policies. From the other end, there is a prevalent use of analytical models to extract information from preemption data (EC2). Our intent is to provide an analogous minimal model for the Google preemptible instances that can then be used to compare with existing analytical models on the same footing (as opposed to comparing simple heurestics in one with analytical forms in the other). This paper addresses the need for a minimal analytical model to describe non-memoryless preemptions, and we expect this to not inhibit the development of simpler empirical modeling approaches, but guide such model development with a fully-parameterized analytical model based on sound first principles and well-defined assumptions.
}

\begin{comment}
\noindent \textbf{Connection to constrained systems and statistical mechanics.} Our proof of Lemma~\ref{lemma:1} used mapping to constrained physical systems and employed the statistical mechanics tools such as partition functions \cite{krauth2006statistical}. 
We have only presented the initial connection between the behavior of constrained preemptions and the statistical mechanics of constraint-driven phenomena in many particle systems \cite{krauth2006statistical,solis}, and we conjecture that a deeper analogy may exist. 
Central to our proof is the assumption of mutually exclusive preemptions---that is, the provider preempts VMs in a mutually exclusive manner.
This assumption makes sense from a cluster management and application perspective. 
However, analyzing constrained preemptions with weaker versions of the mutual exclusion assumption is \emph{also} possible with statistical mechanics approaches. 
For example, for studies of situations where weakly overlapping preemptions are preferred, one can leverage the statistical mechanics framework of constrained ``soft'' particles often investigated using molecular dynamics simulations \cite{jing2015ionic}.
\end{comment}

%where mutual exclusion is only preferred and not mandatory, we can leverage the statistical mechanics of constrained ``soft'' particles, which is also a well studied \cite{solis}, often with molecular dynamics simulations \cite{jyto}. 

%%% Local Variables:
%%% mode: latex
%%% TeX-master: "paper"
%%% End:
