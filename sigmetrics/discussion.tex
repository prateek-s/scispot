\section{Discussion and Future Directions}

We have made an initial exploration into constrained preemptions, and many questions remain to be answered.
Below, we discuss some aspects of pertaining to our model and avenues for future investigation.

\noindent \textbf{What if preemption characteristics change?} Long-term validity of the model cannot be guaranteed since preemption characteristics are up to the cloud provider policies, the supply and demand of transient and on-demand and reserved VMs, etc.
However we note that VMs with constrained preemptions are an interesting \emph{new} type of transient resources. 

\noindent \textbf{Piece-wise model.} Our Statistical analysis indicates that the preemption rates have three distinct phases.
Our model is a continuously differentiable and allows capturing the three phases reasonably. 
However it may be possible to use a piece-wise model, where the three phases are modeled either as three segmented linear regions (found using segmented linear regression), or an initial exponential phase and two linear phases.
Such a piece-wise model would be able to capture the phase transitions with even more accuracy, and is part of our future work. 

\noindent \textbf{Statistical mechanics connection.} We have only presented the initial connection between constrained preemptions and statistical mechanics, and we conjecture that a deeper connection may exist. 
Central to our connection is the assumption of mutually exclusive preemptions---that is, the provider preempts VMs in a mutually exclusive manner.
This assumption makes sense from cluster manager and application perspective. 
However,  analyzing constrained preemptions after relaxing the mutual exclusion assumption is \emph{also} possible with statistical mechanics approaches.
In situations where mutual exclusion is only preferred and not mandatory, we can leverage the statistical mechanics of ``soft'' particles, which is also a well studied and analytically tractable problem. 

%%% Local Variables:
%%% mode: latex
%%% TeX-master: "paper"
%%% End:
