
% \vspace*{\subsecspace}
% \section{Conclusion}
% \label{sec:conclusion}
% The effective use of transient computing relies on understanding the preemption characteristics.
% While past work on transient computing has developed techniques and systems for Amazon's EC2 spot instances, ours is the \emph{first} work to understand the behavior of Google's Preemptible VMs, that have a unique characteristic of having a maximum 24 hour lifetime.
% Our large-scale empirical study shows that the constraint imposes a bathtub failure distribution, and we develop a new preemption probability model for capturing its three distinct temporal phases. 
% Our insights and model-based policies can reduce the preemption overheads by more than $5\times$ compared to existing preemption models, and our batch computing service can reduce computing costs by over $5\times$. 

\noindent \textbf{Acknowledgments.} We wish to thank all the anonymous reviewers and our shepherd Ali Butt, for  their insightful comments and feedback.
%Stephen Lee and Prashant Shenoy's comments on earlier drafts of this paper proved invaluable. 
% This research was supported by NSF grants 1763834, 1802523, 1836752, and 1405826, and Amazon AWS cloud credits.
This research was supported by Google cloud credits for research. 
V.J. was partially supported by NSF through Award DMR-1753182.
%Prateek Sharma was supported in part by a startup grant from Indiana University. Vikram Jadhao was partially supported by the National Science Foundation through award DMR-1753182.


%Constrained preemptions are a relatively unexplored phenomenon and challenging to model. Our model and the associated data expand transient cloud computing to beyond EC2-spot. We have evaluated the model under different practical conditions including different VM types and temporal domains, and have shown it to be general and robust. 


%%% Local Variables:
%%% mode: latex
%%% TeX-master: "paper"
%%% End:
