
\section{Application Policies For Constrained Preemptions}

For applications running on transient servers, the effects of preemptions can be ameliorated through different policies for fault tolerance and resource management.
Prior work in transient computing has established the benefits of such policies for a broad range of applications. 
We now show how our empirical insights and analytical model can be used to develop policies for optimized execution of batch applications on Google Preemptible VMs. 

%We propose to develop optimized preemption mitigating policies that make fundamental use of insights from our models and highlight their practical significance. 

%Our goal is to develop a cost-optimizing execution platform for  batch applications, especially MPI-based scientific computing applications [6] whose large computing requirements and disruption-tolerance make them an ideal application for Preemptible VMs. 
%In particular, the bathtub and time-dependent failures of Preemptible VMs require new policies for these fault-tolerance and resource management problems: 


%\noindent \textbf{High-level System Architecture:} 


\subsection{Job Scheduling}

Many cloud-based applications and services are \emph{long-running}, and typically run a continuous sequence of tasks and jobs on cloud servers. 
In the case of deadline-constrained bathtub preemptions, applications face a choice: they can either run a new task on an already running server, or relinquish the server and run the task on a \emph{new} server. 
This choice is important in the case of non-uniform failure rates, since the job's failure probability depends on the ``age'' of the server.
Furthermore, since newly launched servers also have high preemption rates (and thus high job failure probability), the choice of running the job on an existing server vs. a new server is not obvious.


Our job scheduling policy uses the preemption model to determine the preemption probability of jobs of a given length $T$. 
Assume that the running server's age (time since launch) is $s$. 
Then, the the probability of failure on the existing server $P_{\text{Existing}} = max(1, F(T+s) - F(T))$.
The alternative is to discard the server and launch a new server, in which case, the failure probability is $P_{\text{New}} = F(T)$.
Depending on the server's age $s$ and the job's running time $T$, we can compare $P_{\text{Existing}}$ and $P_{\text{New}}$, and run the job on whichever case yields the lower failure probability.

% Some more intuitive analysis ? 

%Server preemptions lead to \emph{wasted work}, since the progress made by the job is lost (in the absence of any fault-tolerance).
%Minimizing work wasted due to preemptions is a prerequisite to minimizing cost, and 


\subsection{Periodic Checkpointing}

A common technique for reducing the total expected running time of jobs on transient servers is to use fault-tolerance techniques such as periodic checkpointing \cite{flint}.
Checkpointing application state to stable storage (such as network file systems or centralized cloud storage) reduces the amount of \emph{wasted work} due to preemptions.
However, each checkpoint entails capturing, serializing, and writing application state to a disk, and increases the total running time of the application.
Thus, the frequency of checkpointing can have a significant effect on the total expected running time.

Existing checkpointing systems for handling hardware failures in high performance computing, and for cloud transient servers such as EC2 spot instances, incorporate the classic Young-Daly periodic checkpointing interval that assumes that failures are exponentially distributed. 
That is, the application is checkpointed every $\tau = \sqrt{2 \cdot \delta \cdot \text{MTTF}}$ time units, where $\delta$ is the time overhead of writing a single checkpoint to disk. 


However, checkpointing with a uniform period is sub-optimal in case of time dependent failure rates, and especially for bathtub failure rates.
A sub-optimal checkpointing rate can lead to increased recomputation and wasted work, or result in excessive checkpointing overhead. 
Intuitively, the checkpointing rate should depend on the failure rate, and our analytical preemption model can be used for designing an optimized checkpointing schedule. 

Let the uninterrupted running time of the job be $T$. Or in other words, $T$ amount of work needs to be performed. 
Let the checkpoint cost be $\delta$.
We seek to minimize the total expected running time or the \emph{makespan}, which is the sum of T, the expected periodic checkpointing cost, and the expected recomputation.

The makespan $M$ can be recursively defined and computed.
Let $M(W, t, i)$ denote the makespan when $W$ length of job must be exectued, $t$ is amount of job already done. %, and $s$ was the starting time (wallclock time) of the job.
We now need to determine when to take the \emph{next} checkpoint, which we take after $i$ steps. Let $E[M^*]$ denote the minimum expected makespan.


\begin{equation}
  \label{eq:m0}
  E[M^*(W, t)] = \min_{0<i\leq W}{E[M(W, t, i)]}
\end{equation}

The makespan is affected by whether or not there is a failure \emph{before} we take the checkpoint: 

\begin{equation}
  \label{eq:m1}
E[M(W, t, i)] = P_{\text{succ}}(t, i+\delta) \cdot E[M_{\text{succ}}] + P_{\text{fail}}(i+\delta, t) \cdot E[M_{\text{fail}}]
\end{equation}

Here $P_{\text{succ}}(t, i+\delta)$ denotes the probability of the job successfully executing without failures until the checkpoint is taken, i.e., from $t$ to $t+i+\delta$. $P_{\text{fail}}(t, i+\delta) = F(t+i+\delta)-F(i+\delta)$ is computed using the CDF, 
and $P_{\text{succ}} = 1 - P_{\text{fail}}$ .



$E[M_{\text{succ}}]$ is the expected makespan when there are no job failures when the job is executing from step $t$ to $t+i+\delta$, and is given by a recursive definition:

\begin{equation}
  \label{eq:msuc}
E[M_{\text{succ}}(W, t, i)] = t+i+\delta + E[M^*(W-i, t+i+\delta)]  
\end{equation}

\noindent Note that the makespan includes the amount of work already done $t+i$, the checkpointing overhead $\delta$, and the expected minimum makespan of the rest of the job.
Similarly, when the job fails before step $i$, then that portion is ``lost work'', and can be denoted by $E[L(t, i+\delta)]$ which denotes the expected lost work when there is a failure during the time interval $t$ to $t+i+\delta$.
A failure before the checkpoint means that we have made no progress, and $W$ steps of the job still remain.
The expected makespan in the failure case is then given by:

\begin{equation}
  \label{eq:mfail}
 E[M_{\text{fail}}(W, t, i)] = E[L(t, i+\delta)] + E[M^*(W, t+i+\delta)]
\end{equation}



In the case of memoryless distribution, $E[L(t, i+\delta)]$ is approximated as $\frac{i+\delta}{2}$.
In our case, we use the failure CDF instead:

\begin{equation}
  \label{eq:exploss}
E[L(t, i+\delta)] = \int_{t}^{t+i+\delta}{xf(x)~dx}   , 
\end{equation}
where $f(x)$ is our probability density function represented by Equation~\ref{eq:failrate}.

Thus we can find the minimum makespan $E[M^*(W, t)]$ by using Equations~\ref{eq:m0}--\ref{eq:exploss}. 
Given a job of length $J$, minimizing the total expected makespan involves computing $E[M^*(J, s)]$, where $s$ is the current age of the server. 
Since the makespan is recursively defined, we can do this minimization using a dynamic programming approach, and extract the job-steps at which checkpointing results in a minimum expected makespan. 




\subsection{VM Selection}

\noindent \textbf{VM-selection} is an important optimization in cloud environments, because VMs have different tradeoffs of cost, performance, and preemption characteristics.
Application performance is affected by the size of the VM (due to network communication and parallel scaling overheads), and the preemption rates. 
We propose to develop cost models for selecting the ``right'' type of  VMs that minimizes the expected job failure probability and cost by using the analytical preemption models. 
%using our model to \emph{compute the expected average lifetime of a VM} enabling us to minimize the impact of preemptions on performance and overall cost.
Initial analysis indicates that careful VM selection can reduce costs by  up to 30\%.  


\begin{comment}
We note that this search is different from conventional speedup plots in which the objective is to determine how well an application scales with increasing amount of resources and parallelism. 
In contrast, we \emph{fix} the total amount of resources allocated to the application's job ($=\mathcal{R}$), and only vary \emph{how} these resources are distributed, which affects communication overhead and hence the performance.
%Weak
We assume that the total resource requirement for a job, $\mathcal{R}$, is provided by the user based on prior speedup data, the user's cloud budget, and the deadline for job completion.  
\end{comment}


Since server selection involves a tradeoff between cost, performance, and preemptions, we develop a model that allows us to optimize the resource allocation and pick the best VM type that minimizes the expected cost of running an application on \sysname. 

\prat{Highlight tradeoffs here. Larger servers better but more preemption.}


%Let $\mathcal{R}$ denote the total amount of computing resources requested for the job. For ease of exposition, let us assume that $\mathcal{R}$ is the total number of CPU cores.
%Furthermore, let $r_i$ denote the ``size'' of the server of type $i$.
%Then, the number of servers of type $i$ required, $n_i = \mathcal{R}/r_i$.
%In what follows, we denote the expectation value of a quantity as $E[\ldots]$.

%\vj{there is some repitition in defining the symbols here which are used before in selection policy; may be this can be moved above. was wondering if we loose clarity by using $T_k$ to denote the running time on configuration $k$ that encodes the pair defined by the combination of server type $i$ - number of servers of type $i$ -- $(i,n_i)$; that is, $k\equiv (i,n_i)$, used as a superindex?}

Let us assume that the cloud provider offers $N$ server types, with the price (per unit time) of a server type equal to $c_i$. 
The overall expected cost of running a job can then be expressed as follows:
\begin{equation}
  \label{eq:e-cost}
\vspace*{\eqnspace}
  E[C_{( i,n_i )}] = n_i\times c_i \times E[\mathcal{T}_{( i,n_i )}].
\end{equation}
Here, $E[\mathcal{T}_{( i,n_i )}]$ denotes the expected turnaround time of the job (accounting for preemptions) on $n_i$ servers of type $i$.
%
This turnaround time depends on whether the job needs to be recomputed because of preemptions, and is expressed as:
\begin{align}
  \label{eq:turnaround}
  \vspace*{\eqnspace}
  E[\mathcal{T}_{( i,n_i )}] &= T_{( i,n_i )} + E[\text{Recomputation Time}].
\end{align}
Here, $T_{( i,n_i )}$ is the base running time of a job without preemptions, which we obtain empirically as explained in the previous subsection.
Since jobs have to be rerun when they fail due to preemptions, the recomputation time is:
\begin{equation}
  \label{eq:recomput}
   E[\text{Recomputation Time}] = \frac{T_{( i,n_i )}}{2} \times P(\text{at least one preemption})
 \end{equation}
 Our expression of the recomputation time is based on the common assumption that jobs will fail at the half-way mark on average~\cite{daly2006higher, bougeret_checkpointing_2011}. 
%
 The probability that at least one VM out of $n_i$ will be preempted during the job execution is:
\begin{align}
  \label{eq:pfail1}
  P(\text{at least one preemption}) &= 1-P(\text{no preemptions}) \\
                                 &= 1-\left(1-P\left(i,T_{(i, n_i)}\right)\right)^{n_i}.
\end{align}

Here, $P(i, T_{(i, n_i)})$ denotes the probability of a preemption of a VM of type $i$ when a job of duration $T_{(i, n_i)}$ runs on it. 
%
It depends on the type of server, and its associated expected lifetime, and is defined as:
\begin{equation}
  \label{eq:pi}
  P\left(i, T_{\left(i, n_i \right)}\right) = \text{min}\left(\dfrac{T_{(i, n_i)}}{E[L_i]}, 1\right),
\end{equation}

\prat{Can replace by CDF based failure prob}

where $E[L_i]$ is the expected lifetime of the VM of type $i$ extracted using the preemption model (Equation~\ref{eq:expected-lifetime}).
%As a first order approximation, the running time $t$ of the job can be chosen as $t=T_{( i,n_i )}$, where the latter is empirically obtained for a given application.
We also assume that the running time of \emph{individual} jobs in a bag ($T$), will be smaller than the expected lifetime of the VMs, otherwise we will see no forward progress since the jobs will always be preempted before completion.
This is a safe assumption, since more than 90\% HPC jobs are less than 2 hours long (Figure~\ref{fig:hpc-vs-scispot} inset), and the expected lifetime of transient VMs is more than 10 hours.
This restriction only applies to individual jobs---\sysname can run large bags of jobs even if their total running time exceeds the VM lifetime by replenishing preempted VMs.
%We again emphasize that $T$ is the running time of an \emph{individual} job, and that \sysname is designed for running large bags of small jobs, and that most HPC jobs are much smaller than the expected lifetimes, as we show in Section~\ref{sec:eval}. 

\prat{Policy is to use older more stable VMs where possible}

% Using Equations~\ref{eq:turnaround},\ref{eq:recomput}, and \ref{eq:pfail1}, the overall expected cost of running a job on transient cloud servers is obtained as:
% \begin{equation}
%   \label{eq:ecfinal}
%   E[C_{( i,n_i )}] = \frac{1}{2}n_i c_i T_{( i,n_i )}\left(3 - \left(1-\dfrac{T_{( i,n_i )}}{E[L_i]}\right)^{n_i}\right).
% \end{equation}

% Equation \ref{eq:ecfinal} shows that the expected cost $E[C]$ is higher for larger number of servers (high $n_i$), while it is reduced if the expected lifetime of the VM is larger (high $E[L_i]$).








%%% Local Variables:
%%% mode: latex
%%% TeX-master: "paper"
%%% End:
