
\section{Conclusion}
\label{sec:conclusion}
Given the rise of transient cloud computing and its use in web services and distributed data processing, it is not the question of if, but when, transient cloud computing becomes a credible and powerful alternative to high-performance computing for scientific computing applications. 
In this paper, we developed principled approaches for deploying and orchestrating scientific computing applications on the cloud, and presented \sysname, a framework for low-cost scientific computing on  transient cloud servers. 
\sysname develops the first empirical and analytical preemption model of Google Preemptible VMs, and uses the model for mitigating preemptions for ``bags of jobs''. 
\sysname's cost-minimizing server selection and job scheduling policies can reduce costs by up to $5\times$ compared to conventional cloud deployments.
When compared to HPC clusters, \sysname can reduce the total job turnaround times by more than $10\times$. 


%%% Local Variables:
%%% mode: latex
%%% TeX-master: "paper"
%%% End:
