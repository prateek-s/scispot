Scientific computing applications are being increasingly deployed on cloud computing platforms.
Transient servers can be used to lower the costs of running applications on the cloud.
However, the frequent preemptions and resource heterogeneity of these transient servers introduces many challenges in their effective and efficient use.
In this paper, we develop techniques for modeling and mitigating preemptions of transient servers, and present SciSpot, a software framework that enables low-cost scientific computing on the cloud. 
SciSpot deploys applications on Google Cloud preemptible virtual machines, and introduces the first empirical and analytical model for their preemptions. 

\noindent SciSpot's design is guided by our observation that many scientific computing applications (such as simulations) are deployed as ``bag'' of jobs, which represent multiple instantiations of the same computation with different physical and computational parameters. 
For a bag of jobs, SciSpot finds the optimal transient server on-the-fly, by taking into account the price, performance, and preemption rates of different servers. 
SciSpot reduces costs by $5\times$ compared to conventional cloud deployments, and reduces  makespans by up to $10\times$ compared to conventional high performance computing clusters.



%Although transient servers can be used to lower the costs of running scientific computing applications on the cloud, their frequent preemptions and resource heterogeneity makes their effective and efficient use challenging. 

%Our policies enable scientific computing applications to run
%

%SciSpot selects the appropriate transient cloud server for an application 
%of ``bags of jobs'' of scientific computing applications.  
%SciSpot's design is guided by our observation that most  scientific computing applications (such as simulations) are deployed as ``bag'' of jobs, which represent multiple instantiations of the same computation with different physical and computational parameters.
%SciSpot can 
%SciSpot reduces costs by $5\times$ compared to conventional cloud deployments, and makespans by up to $10\times$ compared to conventional HPC clusters.



%Treating bags of jobs as a unit of execution enables simple and powerful policies for optimizing the cost, makespan, and ease of deployment. 
%SciSpot uses Google Cloud Preemptible VMs, and provides the first empirical and analytical model for their preemptions. 
%SciSpot reduces costs by $5\times$ compared to conventional cloud deployments, and makespans by up to $10\times$ compared to conventional HPC clusters.


% We show that optimizing across an entire bag of jobs and being cognizant of the relation between different jobs in a bag, can enable simple and powerful policies for optimizing cost, makespan, and ease of deployment. 


% present SciSpot, a software framework for running scientific applications on transient cloud servers.


% Transient cloud servers can yield significantly lower costs compared to traditional on-demand cloud servers. 
% However, due to their preemptible nature and heterogeneity, their effective use remains challenging.
% %
% % 
% We perform a large-scale, first-of-its-kind empirical measurement and analytical modeling involving over a thousand Google preemptible VMs. 
% Our policies for tackling the resource heterogeneity and transient availability of cloud VMs build on a key insight: most scientific computing applications are deployed as ``bag'' of jobs, which represent multiple instantiations of the same computation with different parameters.
% SciSpot yields a cost saving of 70\% compared to traditional cloud deployments, and a makespan reduction of 20\% compared to a conventional HPC clusters. 

%%% Local Variables:
%%% mode: latex
%%% TeX-master: "paper"
%%% End:
