\section{Design}

Scientific simulation applications consume a large amount of computational resources, and are often used in the context of \emph{exploratory} research, where a large amount of jobs are run with  different simulation parameters.
This can be either a \emph{parameter sweep}, where a large number of parameters need to be evaluated, or a \emph{search} over a large parameter space for the ``right'' set of parameters that yield the desired model behavior.

In this paper, we look at the problem of running scientific simulations on \emph{transient} computing resources in public clouds. 


Past work has largely been focused on running parallel jobs (such as MPI) in the cloud. 
However, considering entire \emph{job-groups} or ensembles of jobs presents new challenges and opportunities in timely, low-cost computation.

Our system, \sysname, is a unified framework for running large job-groups that result from parameter exploration. 

Input and some assumptions:
We assume that the job-group consists of $J_1\ldots J_N$, with each job evaluating a model on some parameter.
The list of parameters to explore can either be generated apriori (as in the case of parameter sweeps), or be dynamically generated as in the case of a search.

In this section, we will look at how we address these challenges:
\begin{enumerate}
\item How to select the right type of cloud server for an application?
\item 
\end{enumerate}



\subsection{Trade-offs in Server Selection}

This presents us with many challenges in the cloud-deployment of these jobs.

Cloud providers offer multiple types of instances (VMs), with different hardware configuration (such as number of CPUs and memory size).
The price of cloud servers is related to their hardware configuration, but it may not be strictly proportional to the hardware performance.
For example, a VM with 32 CPUs may not be 32 times the cost of a single CPU VM.


For parallel and distributed applications, the type of servers selected has large implications on their performance.
Consider the case of deploying an application on 8 8-core VMs vs. 16 4-core VMs.
In both cases, the total number of CPU cores is the same.
However, the larger number of VMs requires more communication between the application tasks, and thus may result in performance degradation.
The performance of applications at different cluster configurations depends on their communication patterns and scaling properties. 



Thus, when deploying applications on the cloud, one has to mindful of the cost and performance tradeoff.
However, in the case of transient servers, the story does not stop here. 


In addition to pricing differences, the transient availability of instances \emph{also} differs by type.
Because the availability of a transient VM is broadly determined by the overall supply and demand of the instances of that \emph{particular} type, the ``preemption rate'' of VMs often depends on the type of the instance.



Thus, selecting a transient cloud server involves a complex tradeoff between the cost of servers, their performance, and the preemption-rate.







%%% Local Variables:
%%% mode: latex
%%% TeX-master: "paper"
%%% End:
