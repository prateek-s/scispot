\section{Related Work}
\label{sec:related}
Our work builds upon a large body of prior work on running scientific applications on the cloud, the various facets of transient cloud computing, and failure modeling. 

\subsection{Scientific Applications In The Cloud}

Running scientific applications on the cloud introduces many tradeoffs compared to conventional HPC clusters, along the dimensions of performance, cost, scalability, convenience, and reproducability.
These tradeoffs are explored in~\cite{iosup_performance_2011, zhai_cloud_2011, marathe2013comparative, galante_analysis_2016, benedictis_cloud-aware_2014}.
In general, clouds can provide increased elasticity, lower waiting times, and more choices in resource allocation that can be tailored to the application.
The cloud resource model is also present in platforms like nanohub~\cite{nanohub}, that provide easy execution and dissemination of nanotechnology simulation applications.
Outside of the bags of jobs execution model, price optimizations for scientific workflows in the cloud is discussed in~\cite{gari_learning_2019}. 



\subsection{Transient Cloud Computing} 

\cite{marathe2014exploiting} classic work on MPI and Spot.
Uses checkpointing.
Redundancy, but for what?
User specified number of VMs. Does not do instance selection. 
BCLR for checkpointing.



MOre spot and MPI:~\cite{gong_monetary_2015}. FOcussed on bidding and checkpoint interval. But bidding doesnt matter. 


\cite{xiang_spotmpi:_2011} is early work for spot and MPI and 

\subsubsection{Server Selection}
Similar to \sysname, SpotOn~\cite{spoton} is also a batch computing service that selects server based on job characteristics and failure rates of different EC2 spot VMs. However, restricted to individual, single-VM batch jobs. 

Heterogenity often used, but not useful in the context of MPI jobs~\cite{exosphere}

Selecting the best instance type, often for data analysis computations~\cite{alipourfard_cherrypick}, and ~\cite{yadwadkar_selecting_2017}, and others like Ernest and Hemingway.


Since servers in the spot market are significantly cheaper than the
equivalent on-demand servers, they are attractive for running
delay-tolerant batch jobs~\cite{spoton, jain14demand, 
  yi2010reducing, conductor, liu-spot, spot-run, dubois2016optispot}. 

Parameter sweep aka bags of jobs ~\cite{casanova_heuristics_2000}

But nothing for bags of jobs themselves.. 


\subsubsection{Fault-tolerance}

All the past work was on EC2 spot market with gang failures and independent markets~\cite{marathe2014exploiting, gong_monetary_2015}.
However this assumption has now changed, and failures can happen anytime.
Our failure model is more general, and applies to both cases.



\cite{dongarra_fault_nodate} has a discussion of checkpointing frequency which is comprehensive. Replication is another way~\cite{walters_replication-based_2009}

Non periodic checkpointing~\cite{bougeret_checkpointing_2011}


\subsubsection{Failure Modeling}

A number of related papers also analyze spot price data to better understand its statistical  characteristics~\cite{bidding4,mihailescu2012impact,bidding7,bidding1,bidding8,bidding3,bidding6,bid-cloud,bidding5}.  


\cite{wolski_probabilistic_2017} \cite{guo_bidding_2015}



Crevecour etc. 



\subsubsection{Server Selection}
Exploring a large configuration space using bayesian optimization methods in CherryPick~\cite{alipourfard_cherrypick} and Metis~\cite{li2018metis}.

Can also use Latin Hypercube sampling for parameter exploration?




%%% Local Variables:
%%% mode: latex
%%% TeX-master: "paper"
%%% End:
