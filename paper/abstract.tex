%In this paper, we...

Given the dramatic rise of cloud computing resources and their utilization in web services and distributed data processing, it is not the question of if, but when, cloud computing becomes a credible and powerful alternative to high-performance computing for scientific computing applications. To accelerate this transition and enable the implementation of scientific computing tasks over the cloud, a comprehensive understanding of the dynamics of preemptions of cloud servers, particulary in the case of transient cloud resources, and a reliable set of associated preemption-mitigation policies that optimize for cost, makespan, and deployment ease are required. 
In this paper, we develop principled approaches for deploying and orchestrating scientific computing applications on the cloud, and present \sysname, a framework for low-cost scientific computing on  transient cloud servers. 
We perform a large-scale, first-of-its-kind empirical measurement study involving over a thousand Google preemptible VMs of different types and present the \emph{first} empirical model of transient server availability. This empirical data informts the formulation of a novel analytical model to describe preemption dynamics that enables the prediction of expected running costs of jobs of different types and duration.
Our policies for tackling the resource heterogeneity and transient availability of cloud VMs build on a key insight: most scientific computing applications are deployed as ``bag'' of jobs, which represent multiple instantiations of the same computation with different parameters.
We show that optimizing across an entire bag of jobs and being cognizant of the relation between different jobs in a bag, can enable simple and powerful policies for optimizing cost, makespan, and ease of deployment, and implement these policies as part of the \sysname~framework. 
The preemption-mitigation policies developed to minimize the overall makespan of bags of jobs, by taking into consideration the partial redundancy between different jobs within a bag, yield a cost saving of 70\%, and a makespan reduction of 20\% compared to a conventional HPC clusters.

\vspace*{10cm}

%%% Local Variables:
%%% mode: latex
%%% TeX-master: "paper"
%%% End:
