\section{Background and Overview}

In this section, we give on overview of the characteristics and challenges of transient cloud computing; motivate the need for the bag of jobs abstraction in scientific computing workflows; and give an overview of our \sysname system. 


\subsection{Transient Cloud Computing}
% The "Environment"

Infrastructure as a service (IaaS) cloud offerings such as Amazon EC2, Google Public Cloud, Microsoft Azure, etc., typically provide computational resources in the form of virtual machines (VMs), on which users can deploy their applications.
% too long an opening sentence, not to the point 
Conventionally, these VMs are leased to users on an ``on-demand'' basis: cloud customers can start up a VM when needed, and the cloud platform provisions and runs these VMs until they are relinquished by the customer. 
Due to the temporal dynamics of cloud workloads, the overall utilization of cloud platforms is also highly dynamic. 
To satisfy user demand, cloud capacity is typically provisioned for the peak load, and thus the average utilization tends to be low, of the order of 25\%~\cite{borg,resource-central-sosp17}. 


To increase their overall utilization, large cloud operators have begun to offer their surplus resources as low-cost servers with \emph{transient} availability, which can be preempted by the cloud operator at any time (after a small advance warning). 
These preemptible servers have become popular in recent years due to their discounted prices, which can be 7-10x lower than conventional non-preemptible servers. 
Cloud offerings such as Amazon Spot instances~\cite{spot-web}, Google Preemptible VMs~\cite{preemptible}, and Azure batch VMs~\cite{azure-batch} are examples of such low-cost transient servers. 


However, effective use of transient servers is challenging for applications because of their uncertain availability~\cite{spotcheck, prateek-thesis}. 
Preemptions are akin to fail-stop failures, and result in loss of the application's memory and disk state, leading to downtimes for interactive applications such as web services, and poor throughput for long-running batch-computing applications. 
Consequently, researchers have explored fault-tolerance techniques such as checkpointing~\cite{flint, marathe2014exploiting, spoton} and resource management techniques~\cite{exosphere} to ameliorate the effects of preemptions for a wide range of applications. 
However, since the effect of preemptions is dependent on a complex combination of application resource and fault model, mitigating preemptions for different applications remains an active research area~\cite{eurosys19-graph}. 






\subsection{Bag of Jobs in Scientific Computing}


\subsection{SciSpot Overview}
Our system, \sysname, is a unified framework for running large job-groups that result from parameter exploration. 



% We will modify our text to safeguard against R2's misunderstanding.
% statistical pattern recognition/machine guessing 