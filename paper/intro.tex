\section{Introduction}

Running parallel scientific applications, such as molecular dynamics (MD) simulations, on low-cost cloud transient resources. 

The first "big" idea is that simulations are often bag of parallel tasks. 

While there has been some past work that looks at running MPI applications on spot instances, our scope is much broader and considers how complete simulation pipelines can be run at low cost. 

Spiel on transient instances. Increasingly popular resource allocation model that is being offered by all cloud providers. 
Very low cost compared to conventional cloud resources, often by up to 10x. 
However, can be frequently revoked. 
Thus failure is a common occurrence, and not a rare-event. 
This is especially challenging for MPI jobs because of its inability to tolerate failures. 

However, our insight is that while protecting a *single* job against revocations can require elaborate checkpointing based approaches, we dont necessarily have to do that if we consider that most simulations are composed of a series of jobs that search over a parameter space, and that what is important is the total running time and cost of this entire series of jobs. 

Thus, no single job is "special". 

Another aspect of novelty is that past work on transient resources used EC2 pricing information to get failure probabilities. However, this is no longer an accurate method. We perform the first empirical study of google preemptible VMs and their performance and availability for HPC workloads. 

Another fundamental question is what is a suitable metric in such cases. Conventionally, it is speedup. In the cloud, it is some combination of cost and running time. 

%%% Local Variables:
%%% mode: latex
%%% TeX-master: "paper"
%%% End:


\sysname is a framework and a tool that combines the use of failure modeling, checkpointing, and application-aware early stopping, to provide low cost execution of jobgroups for scientific applications.


Our work is the first to make a principled study of transient instances \emph{other} than Amazon spot instances.
Furthermore, our techniques make the first stab at addressing the new problems in the new EC2 spot pricing scheme.

Our work is in the context of reliability and cloud execution of scientific applications, and is novel because of multiple reasons:
1. Reliability and failure analysis of parallel scientific applications usually studied in the context of hardware with MTTFs of centuries, which is several orders of magnitudes higher than MTTFs faced in transient cloud servers (few hours).

2. While there have been studies of scientific applications been deployed in the context of cloud platforms, to the best of our knowledge, there has been no effort that integrates server selection and running jobgroups in a convenient automatic manner that makes it feasible to actually deploy applications on the cloud for scientists who may not have the requisite cloud experience.


\begin{comment}
Notes:
  
For scientific applications, public clouds offer many advantages such as no waiting/queuing time and instant access to a wide range of resources, and pay as you go pricing.
However, judicious use of cloud resources is necessary to achieve high performance and to avoid cost overruns.

Increasingly, cloud providers are offering transient VMs that are sold at steeply discounted rates of 90\%.
However, they can be unilaterally revoked by the cloud provider, resulting in preemptions which are akin to fail-stop failures.
This is 



\end{comment}