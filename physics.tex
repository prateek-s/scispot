\subsection{Generalization and Physical Interpretation}

In this subsection, we seek a broader understanding of constrained preemptions by looking at other scenarios where our model may be applicable, and show a surprising connection with statistical physics. 
We saw earlier that the constraint of finite lifetime exhibits a bath-tub shaped preemption distribution, and seek more insight into this behavior. 

%fundamental reasons as to why this behaviour might be expected. 


While the exponential aging/reclamation model is a good empirical  model for constrained preemptions, we ask the question: ``Are there underlying processes that can explain the bath-tub characteristics?''
While the preemption behaviour is ultimately up to the cloud provider, we postulate some general observations that are based on insights and theories from statistical physics. 


From a cluster management perspective, preempting a VM entails identifying and marking the victim VM according to some policy, giving it the preemption warning, and finally terminating it. 
This process frees up cluster resources and causes a change in the overall resource allocation state of the cluster. 
These preemption operations performed on different VMs may be carried out in a mutually exclusive manner so that only one VM at a time is preempted, and the cluster state information is consistent. 


This ``only one preemption at a time'' policy might also be be helpful to applications (and therefore cloud users) that want to avoid concurrent preemptions for reliability or performance reasons.
For instance, for a clustered web application, losing a single server (due to preemption) out of $n$ servers will mean that the load can be distributed among the remaining $n-1$ servers. 
Assuming $k$ servers are preempted simultaenously, the increase in load is $\frac{n-k}{n}$.
If the application's preemption-handling policy requires replenishment, this increase in load is temporary, but may still lead to SLA latency violations. 
% Needs explanation. 
Thus, we can assume that the preemption operation operations cannot overlap.


\noindent \textbf{From Preemptions To Particles.}
Given this assumption of non-overlapping preemptions, we now seek to model the probability distribution of preemption events. 
Our main insight, is that this is analogous to finding the distribution of $N$ randomly distributed  particles of diameter $d$ in a 1 dimensional interval of length $L$. 
Just as preemption events cannot overlap, these particles cannot occupy the space of another particle (i.e., they are ``hard''). 


The problem of analyzing distributions of particles in such a setup has long been studied in statistical physics. 
\emph{Surprisingly, the distribution of particles in a finite interval is not uniformly distributed. 
  Instead, particles are more likely to be found near the two end-points of the interval.}
This is similar behaviour as observed with the preemption data: preemptions are more likely to occur near the start and end of the interval. 


We show the particle distribution in Figure~\cite{fig:particle-sim}, which also exhibits a sort of bath-tub behavior. 
The probability distribution has been extensively studied, and closed-form solutions are known for various physical configurations.
In particular, near the edge, the probability of finding a particle is equal to
\begin{equation}
  \label{eq:balls-krauth}
  P(t\geq s) = \dfrac{1}{L-Ns}\left(1-\dfrac{N-1}{L-Ns}(t-s/2)\right)
\end{equation}

%This explanation has turned out to be too similar to Krauth. Needs some fresh words. 
Informally, this occurs because each particle has a ``halo'' around it in which other particles can't be around. 
When placed near the edge, the particle's halo is reduced---leaving more space for other particles, and thus this configuration has higher probability if particles are placed occur randomly. 
This can also be used to explain why particles tend to cluster together, because their halos overlap and the ``cancel'' each other. 


In this model, the ``size'' of a preemption is the duration of the temporal mutual exclusion; the number of particles is the number of preemptible VMs under the cluster management perspective; and the length is the maximum VM lifetime (i.e., 24 hours). 

\noindent \textbf{Temporally Constrained Dynamics.}
The particle analogy allows us to analyze other situations in which non-overlapping events must occur in a confined interval. 
The theoretical and practical analysis of particles in a confined environment, provides us a strong conceptual framework with which to analyze such scenarios. 
Our goal here 


% Why are we doing this?


%%% Local Variables:
%%% mode: latex
%%% TeX-master: t
%%% End:
